\documentclass{article}
\usepackage[utf8]{inputenc}
\usepackage{graphicx}

\title{Memory}
\author{MCB C61 with Professor David Presti \\ \\ Benjamin Lee}
% \date{8 March 2018}

\begin{document}

\maketitle

\textbf{Key Concepts:}
\begin{itemize}
    \item pi
    \item \textit{The Mind of a Mnemonist}
    \item Synesthesia
    \item Memory: working or short-term, long-term
    \item Consolidation
    \item Amnesia: retrograde, anterograde
    \item Dementia: Vascular, Alzheimer's
    \item Drugs and meomry impairment
    \item Nootropic drugs
    \item Karly Lashley
    \item Donald Hebb
    \item Patient H.M.
    \item Hippocampus
    \item Memory: declarative, non-declarative
    \item \textit{Aplysia Californica}
    \item Eric Kandel
    \item Gill-withdrawal learning
\end{itemize}



\newpage
Lecture Notes:

Working Memory (WM) (Short Term Memory) (STM)
limited capacity transient

Long Term Memory (LTM)
initially fragile
consolidation (neural plasticity??) (from STM to LTM) (Hippocampus to long term storage somewhere) (sleep is key to consolidation) 
structural change (not localized, distributed)

Karl Lashley (1890 - 19558) 
rat maze in 1920's
cut up the rats brain


H.M. Henry Molaison (1926-2008)
severe seizures since age 10, surgery at age 27, 1953
Removed his hippocampus because his seizures were thought to be amplified by the hippocampus, which distributes info throughout the brain

WM ~okay
old LTM ~okay
could not learn new information
hippocampus as hub of distributed storage and consolidation

declarative  

nondeclarative
ex: riding a bicycle

Structural change inn neural networks 
Danald Hebb (1902-1985) 
Hebbian learning (neuroplasticity) 
Aplysia californica (sea slug, sea hare) 
shock siphon or gill of sea slug
Following a single aversive stimulus to tail, synaptic strengthening lasts ~ one hour (STM) 
Found K$^+$ leak channels 




\end{document}