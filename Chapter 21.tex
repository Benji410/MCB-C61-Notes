\documentclass{article}
\usepackage[utf8]{inputenc}
\usepackage{graphicx}

\title{Emotion}
\author{MCB C61 with Professor David Presti \\ \\ Benjamin Lee}
% \date{8 March 2018}

\begin{document}

\maketitle

\textbf{Key Concept:}
\begin{itemize}
    \item Feeling, emotion, mood
    \item Darwin and emotions
    \item Facial expressions
    \item Paul Ekman: constuctivism vs. evolution
    \item Vagus Nerve
    \item Amygdala, hypothalamus
    \item Oxytocin, vasopressin
    \item Reward pathways, Jame Olds, dopamine
    \item Serotonin, mood disorders, depression
    \item Clinical antidepressant medications
    \item Pro-social emotions
    \item Mindfulness meditation
\end{itemize}

\newpage

\section{Feeling, Emotion, Mood}
In the characterization of mental experiences as thoughts, feelings, and perceptions:
\begin{itemize}
    \item Thoughts are considered to have a linguistic aspect, representable as either words in a kind of subjective inner dialogue or images 
    \item Perceptions have a direct sensory quality, such as shape or color, tones, taste and aroma, hot and cold, etc
    \item Feelings, in contrast, have a kind of nonlinguistic and intuitive quality to them 
\end{itemize}

\noindent \textbf{Feelings} are considered to be the mental experience component of \textbf{emotions}, such as joy, anger, fear, surprise, sadness, and disgust. 
\begin{itemize}
    \item Emotions experienced in the entire body, not just mind. 
    \item Outward signs associated, such as facial expressions, body posture, laughter, tears, and changes in heart rate, blood pressure, breathing, and skin temp. 
\end{itemize}

\noindent \textbf{Emotions} have profound and direct impact on behavior, on action. Prepare us for and move us to action out into the world, in immediate and powerful ways

\begin{itemize}
    \item Emotion and consciousness have a strong association
    \item \textit{Sentience:} synonym for conscious awareness, which means to feel
    \item \textbf{Rene Descartes} famously wrote: \underline{"I think, therefore I am"} - "I feel, therefore I am"
    \item Our experience of who we are is grounded in "what it is like to be" as a whole interacting with the worl
    \item Emotions are unbidden and spontaneous, reveal private aspects of our inner experience, social communication
    \item Emotions like happiness, joy, compassion, tenderness, admiration, awe, affection, and love hlep us get along and form relationships with others; they are at the foundation of cooperative social systems
\end{itemize}

\noindent \textbf{Moods} are more prolonged whereas emotions may be brief. Sometimes used interchangeably 
\begin{itemize}
    \item Emotions may be triggered by events but moods are not the same case
    \item \textit{mood} related \textit{mode} meaning manner, method, or way
    \item Emotions experienced over time may become or contribute to, moods
    \item Moods of some kind are always present, at some level, providing a continuous background of feeling to our experience 
\end{itemize}
Emotions and moods are often characterized as being either positive or negative (pleasant, enjoyable vs. upset or distress)

\section{Darwin and Emotions}
Charles Darwin was a pioneer in the science of emotion
\begin{itemize}
    \item \textit{The Expression of the Emotions in Man and Animals} published in 1872
    \item Put emotions into an evolutionary context, arguing that emotions are not unique to humans but evolved as adaptive behaviors that we share with other animals
    \item Made connections between animal expressions and human expressions
\end{itemize}

Mental experiences of emotions and feelings in nonhuman species remains a debated topic
\begin{itemize}
    \item Some say language is necessary to produce mental experience
    \item Some argue mental experience is still there without language
    \item Human language adds nuance and complexity to those mental experiences, but not essential for awareness
\end{itemize}

Darwin argued for a universality of emotions, a continuity of expression extending at least from our primate relatives to us humans

\subsection{Constructivism vs Evolution}
\textbf{Constructivism:} Understanding human behavior largely in terms of cultural influences and determinants. Most basic aspects of emotional expression depend largely on cultural factors. \\
Ex: a smile may be interpreted differently from one culture to another \\

\noindent \textbf{Evolution:} Darwin's view implying that expression of emotion, and the interpretation of emotional expression, would be universal among humans, across people from many different cultures \\

\noindent \textbf{Paul Ekman} tested constructivist vs evolutionary perspective on emotion in 1960's using facial expressions
\begin{itemize}
    \item Facial expressions associated with several emotions (joy, anger, fear, surprise, sadness, disgust) were universal across many human societies
    \item Including tribal people in New Guinea who had minimal contact with the other cultures
    \item Darwin would have predicted these results
\end{itemize}

\noindent There are deep biological determinants to the experience and expression of human emotion. \\
There are cultural factors as well. \\ 
Ex: Ekman demonstrated culture-specific "display rules" that "specify who can show which emotion to whom and when."

\subsection{Facial Expressions}
Darwin's discussion of human facial expressions drew from French neurologist Guillaume Duchenne (1806-1875)
\begin{itemize}
    \item Studied muscular control of facial expressions by selectively activating specific muscles using direct electrical stimulation
    \item Zygomaticus muscle lifts the corners of the mouth when we smile. 
    \item Spontaneous smiles - ones associated with true enjoyment
    \item Also involve activation of the orbicularis oculi muscles around the eyes
    \item Ekman called this smile the Duchenne smile, associated with pleasure, joy, or enthusiasm
\end{itemize}

\noindent \textbf{Facial expressions} are one aspect of body signatures associated with \textbf{emotions}
\begin{itemize}
    \item Facial expressions function primarily to communicate emotions to other
    \item Other body signatures include changes in heart rate, blood pressure, skin temperature (autonomic nervous system effects), tone of voice and body posture, hormonal release into blood circulation (neuroendocrine effects) such as cortisol and adrenaline from adrenal glands and oxytocin from hypothalamus (pituitary gland) 
\end{itemize}

\noindent \textbf{Hormone:} chemical release by endocrine glands into blood circulation, whereupon they mediate effects throughout the body 

\newpage
\section{Vagus Nerve}
Plays important role in autonomic neural component of emotion 
\begin{itemize}
    \item Cranial Nerve 10 
    \item Consists of neural connections between the brain and large region of  the core of the body (Latin \textit{vagus} = wandering) 
    \item Included in circuitry is parasympathetic innervation of the body's core - the heart, lungs, digestive system, and other parts
    \item Parasympathetic input to the heart functions to decrease hear rate (Otto Loewi, frog hearts CH6) 
\end{itemize}

\noindent \textbf{Vagus nerve complexity:} 
\begin{itemize}
    \item Fiber bundle consists of many axons, some myelinated and some not
    \item \textbf{Efferent Fibers:} Carries signals from the brain to body organs 
    \item \textbf{Afferent Fibers:} Carries signals from interior of the body to the brain
    \item 50 percent of the axons in vagus nerve are afferent fibers
    \item "gut feelings": emotional experience involves awareness of physiological state of the internal core of our body (related to robust communication in both directions)
    \item Vagus nerve activity associated with more relaxed emotional style (equanimity: mental calmness)
    \begin{itemize}
        \item Resilience when encountering negative emotions
        \item More frequent experience of positive emotions, greater prosocial expression (empathy, social connection) 
        \item Improvements in physical health
    \end{itemize}
\end{itemize}

\noindent \textbf{Amygdala:} a group of nuclei at the base of the temporal lobes and heavily interconnected with sensory areas of the cerebral cortex
\begin{itemize}
    \item Also connected with  groups of neurons in brainstem
    \item Involved in the perception of emotional expressions
    \item Involved in signaling the hypothalamus to initiate a cascade of events that forms part of the body's response to stressful events
\end{itemize}

\noindent\textbf{Hypothalamus:} produces neuropeptides that regulate the release of systemic hormones from the adjacent pituitary gland
\begin{itemize}
    \item One hypothalamic-pituitary neural connection regulates release of adrenocorticotropic hormone from the pituitary gland into blood circulation
    \item Hormone triggers adrenal gland to release \textbf{cortisol}, a steroid hormone that increases the availability of glucose to cells
    \item Cortisol is part of our body's response to perceived threat and other kinds of stress
\end{itemize}

\noindent Other neurons in the hypothalamus produce the neuropeptides \textbf{oxytocin} and \textbf{vasopressin}: 
\begin{itemize}
    \item Two molecules are chemical relatives, consist of polypeptide chain of nine amino acids
    \item \textbf{Oxytocin:} Cys-Tyr-\textbf{Ile}-Gln-Asn-Cys-Pro-\textbf{Leu}-Gly
    \item \textbf{Vasopression:} Cys-Tyr-\textbf{Phe}-Gln-Asn-Cys-Pro-\textbf{Arg}-Gly
    \item Only differ by the two bolded amino acids
    \item Both produced by neuron in hypothalamus and released as hormones via the adjacent pituitary gland into the blood circulation
\end{itemize}

\noindent \textbf{Oxytocin} act on the female uterus during childbirth to induce contractions and facilitate birth \\
Also stimulates the production and release of milk from the mammary glands \\

\noindent \textbf{Vasopressin} acts on the kidneys to slow the transfer of water from the blood to the urine and also acts systemically to constrict blood vessels

\noindent \textbf{Other effects of Oxytocin and Vasopressin:}
\begin{itemize}
    \item Act at sites in brain by GPCRs that respond to either hormone
    \item Don't reenter brain (BBB) but enter cerebrospinal fluid to affect CNS
    \item Axons from hypothalamic neurons release oxytocin and vasopressin into the brainstem, hippocampus, amygdala, nucleus, accumbens and other regions interior of brain
    \item Important in mother-infant bonding
    \item Increase prosocial behaviors in those suffering from autistic spectrum disorders, depression, and schizophrenia
\end{itemize}

\newpage
\section{Reward Pathways}
"Feeling good" often associated in popular media with brain "reward pathways" 

\subsection{James Olds (1922-1976) }
1950's discovered regions of the brain that, when electrically stimulated, the rats wanted more electrical stimulation. 
\begin{itemize}
    \item Perhaps electric stimulation interpreted as rewarding or pleasurable to the animal
    \item Originally dubbed "pleasure centers" 
    \item Wired neural pathway to a lever and had rats press a level to activate stimulus
\end{itemize}

\subsection{Dopamine}
\noindent Best-known reward circuit involves \textbf{dopaminergic} neurons in the \textbf{ventral tegmental} area projecting to the \textbf{nucleus accumbens} and \textbf{prefrontal cortex} areas. 
\begin{itemize}
    \item All three areas are interconnected with the hypothalamus, amygdala, and hipocampus
    \item Pathway activated in connection with "enjoyable" experiences: listening to music, eating chocolate, ingesting drugs, and orgasms
    \item Dysfunctional compulsive behaviors (addictions) disrupt the normal activity of this circuit
\end{itemize}

\noindent\textbf{Limbic system}: structures along the interior border of the cerebrum, at the ege between the cerebrum and the brainstem \\

Limbic and prefrontal cortex are key parts of the emotion regulation circuitry 

\subsection{Serotonin}
Also has role in regulating emotions, like oxytocin and dopamine. 
\begin{itemize}
    \item Assigned roles to particular transmitters
    \item Oxytocin: cuddle hormone, love neuropeptide, compassion molecule
    \item Dopamine: pleasure neurotransmitter
    \item Serotonin: molecular mediator of positive mood
\end{itemize}

Serotonin-mood connection derives in large part from neurochemistry of drugs used to treat \textit{mood disorders} \\

\noindent \textbf{Mood disorder:} stuck in a prolonged pepriod of time in a mood state that significantly interferes with one's ability to function and flourish in life 

\begin{itemize}
    \item \textbf{Depression:} (clinical depression, major depression) prolonged and dysfunctional dysphoric mood, a malignant melancholia
    \item \textbf{Mania:} prolonged and dysfunctional euphoric mood (euphoria accompanied by poor judgement and gradiosity) (opposite of depression) 
    \item \textbf{Manic-depressive disorder (bipolar)} characterized by periods of both mania and depression
    \item \textbf{Anxiety disorders:} prolonged anxious moods, chronic, protracted manifestations of nervousness that interfere with one's ability to function
\end{itemize}

\subsection{Clinical Antidepressant Medications}
First two categories of pharmaceutical antidepressants in 1960s
\begin{itemize}
    \item \textbf{Monoamine oxidase inhibitors (MAOIs):} inhibits enzyme monoamine oxidase that normally inactivates norepinephrine and serotonin by oxidizing them 
    \item \textbf{Tricyclic antidepressants (TCAs):} blocks or slows the reuptake of norepinephrine and serotonin back into the axons of the neurons that release them 
    \item Both increased presence of neurotransmitters norepinephrine and serotonin at synapses
    \item From 1950s to 1980s, MAOIs and TCAs used to treat clinical depression and main understanding for treating depression and mood disorders
\end{itemize}

\noindent 1988, fluoxetine Prozac hit market
\begin{itemize}
    \item Selective serotonin reuptake inhibitors (SSRIs): Inhibits serotonin reuptake receptors
    \item Other drugs: sertraline (Zoloft), paroxetine (Paxil), citalopram (Celexa)
    \item No more effective than MAOIs and TCAs, but less harmful side effects
\end{itemize}

With the heavy advertisement of these SSRIs being perceived as nontoxic, "nondepressed" people began taking these drugs as well as antipsychotics. 

\section{Prosocial emotions and "mindfulness"}
A prominent tradition in theorizing about the psychology of emotion is that the human condition is one of fear and dread, and the dominant human emotions are negative ones related to states of fear, anxiety, anger, pain, suffering , and desire. In this view, positive emotions are considered secondary to these negative states and emerge only with the occasional cessation of the negatives states. \\
Meaning humans are naturally negative, we are selfish and this draws from our innate beings. "survival of the fittest", there is no room to be positive in this blood thirsty selfish world. \\ 

Dacher Keltner in his 2009 book \textit{Born to Be Good} presents a thesis that prosocial emotions are what matter most in human behavior. We excel at laughter, play, love, gratitude, compassion and forgiveness. Although human conflict is pervasive, humans also have highly refined emotional abilities to preempt and to resolve conflict. \\ 

"Mindfulness meditation": training and practice in bringing one's awareness repeatedly back to a focus of attention, such as the breath, has been associated with manifestations of prosocial emotion and emotional balance, decreases in measures of depression and anxiety, decreased perceived stress, and enhanced functioning of the immune system. 

\end{document}