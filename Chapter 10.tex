\documentclass{article}
\usepackage[utf8]{inputenc}

\title{Neural Development and Neuroplasticity}
\author{MCB C61 with Professor David Presti \\ \\ Benjamin Lee}
\date{March 2018}

\begin{document}

\maketitle

\textbf{Key Concepts}
\begin{itemize}
    \item Human genome: size, \% transcribed, \% translated
    \item Transcription factors
    \item Stem cells, cell differntiation
    \item Neural progenitor cells
    \item Neural tube
    \item Growth cone
    \item Cytoskeleton, microtuble, microfilament, actin, tubulin
    \item Roger Sperry
    \item Chemoaffinity hypothesis
    \item Neurotrophins, nerve growth and guidance factors
    \item Pruning 
    \item Neuroplasticity
    \item Embryonic and adult neurogenesis
    \item Hipocampus
\end{itemize}
\newpage

\section{Human Genome}
Human Genome consists of fort-six chromosomes: twenty-three inherited from the mother and twenty-three inherited from the father. \\
46 = 23 (Dad) + 23 (Mom) \\ 
$X$ from the mother and $X \ \text{or} \ Y $from the father. \\
\subsection{Size}
\textbf{haploid:} 23 or half the number of chromosomes \\
\textbf{diploid:} 46 or the amount of chromosomes that a human should have. \\
Twenty-three chromosomes of the haploid human genome contain approximately three billion $(3  *  10^9)$ nucleotide base pairs: adenines, thymines, guanines, and cytosines. \\
Entire human genome has been sequenced, every pair and the order they are in is known. \\
Although everyone contains the same genome, they are transcribed and translated differently in different cells. \\
\subsection{Percent (\%) Transcribed \& Percent (\%) Translated}
Of the Human Genome: 
\begin{itemize}
    \item less than 3 percent codes for functional protein
    \item more than 85 percent is transcribed into RNAs of various kinds
    \item RNAs not translated into proteins appear to be crucially involved in the regulation of gene expression, although not fully understood
    \item greater than 97 percent of the human genome that does not code for functional protein is dubbed as the dark matter of DNA. 
\end{itemize}

\textb{Transcription factors:} proteins that bind to regions of the DNA and regulate the readout of genes

\section{Stem Cells}
Embryonic stem cells: after conception, the fertilized egg begins to divide and form multiple cells, with \textbf{the capacity to continue dividing and to differentiate into any type of cell in the body.} \\ 
Cell differentiation: when less specialized cells become more specialized, hence stem cells. \\
Embryonic stem cells in the developing nervous system differentiate into \textbf{neural progenitor cells} \\
\textbf{Neurogenesis \& Gliogenesis:} formation of nerve cell and glial cells from neural progenitor cells \\
\textbf{Migration:} as cells differentiate, they move around to occupy specific locations \\
\textbf{Synaptogenesis:} as neurons mature they also begin to wire together, forming synapses \\
Within three weeks of conception in a developing human embryo, a group of cells begins to fold and form a structure called the \textit{neural tube}. The entire CNS will develop from this folded piece of tissue as it grows and differentiates. \\
\subsection{Neurogeny: Growth of an Embryo's Mind}
Human conception to human birth requires approx. 9 months of development in the womb 
\begin{itemize}
    \item Three Weeks: \textit{neural tube} is formed
    \item Third Month: Brain growth takes off. 
    \item Seventh Month: Convolutions of cerebral cortex, gyri and sulci, begin to form. Reflecting rapid expansion of the density and connectivity of cells in the cerebrum. 
\end{itemize}

\subsection{Growth Cone}
"living battering ram, soft and flexible, which advances, pushing aside mechanically the obstacles which it finds in its way, until it reaches the area of its peripheral distribution" -Ramon y Cajal 
\begin{itemize}
    \item Progresses via the extension of fingerlike structures called \textit{filopodia}
    \item This extension as well as the migraiton of entire cells as they find their appropriate places in the developing organism is propelled by actions of the internal \textit{cytoskeletal} structure of the cell
\end{itemize}
\textbf{Cytoskeletal:} The dynamic cytoskeleton is composed of elaborate ordered arrays of protein polymers: 
\textbf{microfilaments} made of \textit{actin} proteins and \textbf{microtubules} made of \textit{tubulin} proteins. \\
Microfilaments and microtubules form long strands that perform multiple functions within cells including: 
\begin{itemize}
    \item Growth and movement of cell processes (axons, dendrites, and dendritic spines) 
    \item Moving materials around within the cell
    \item Insertion and removal of membrane proteins (ion channels, transporters, and neurotransmitter receptors)
\end{itemize}

\section{Roger Sperry (1913-1994)}
1930s and 1940s conducted clever investigations of the process by which neurons form connections
\begin{itemize}
    \item Research with amphibians with remarkable capacity to regenerate after sustaining physical damage to their bodies. (frogs and salamanders)
    \item Severed frogs optic nerve. Several weeks later, the optic nerve regenerated, completely restoring the frogs vision. (Severe a human optic nerve, it doesn't regenerate) (Regrew from eye to frogs brain)
    \item "How did axons from the eye know where to form synapses in the brain, so that normal vision is restored?"
    \item Experimented with rotating frog eyeball and severing optic nerve. 
\end{itemize}
\subsection{Chemoaffinity Hypothesis}
Sperry proposed that nerve cells use specific chemical signals to guide their wiring during development and during neural regeneration. \\
\textbf{Nerve growth and guidance factors:} variety of protein molecules and mechanisms that regulate the processes of cell growth, differentiation, migration, and synaptogenesis. \\
\textbf{Neurotrophin:} \\
\indent First nerve growth factor, name "nerve growth factor" or NGF \\
\indent Found more factors no called BDNF (brain-derived neurotrophic factor), GDNF (glia-derived neurotrophic factor, and NT3 (neurotrophin-3) \\
Other proteins, involved in axon and dendrite guidance, as well as other development processes: ephrin, netrin, neuropilin, plexin, semaphorin, Slit and Robo. 
\section{Neuroplasticity}
Synapses that are used become stabilized and strengthened \\
Synapses that are not used are eliminated in a process called \textbf{synaptic pruning (pruning)} \\
May happen as a result of molecular process taking place either presynaptically, postsynaptically, or both. \\
\textbf{Presynaptic strengthening and weakening:}
Glutamatergic synapse, glutamate neurotransmitter may interact with glutamate receptors located on the presynaptic axon terminal to open Na$^+$ or Ca$^+$ ion channels and thus prolong depolarization in the axon terminal, thereby strengthening the synapse. \\
Addition of more reuptake transporter proteins means neurotransmitter is removed from the synaptic cleft more rapidly after release, thus a smaller signal, weakening a synapse. \\
Less transporter proteins = stronger synapse \\
\textbf{Postsynaptic strengthening and weakening:} 
More neurotransmitter receptors = greater impact from incoming neurotransmitter (strengthening of synapse) \\
Fewer postsynaptic neurtransmitter receptors mean a weaker synapse \\

\section{Hippocampus}
Bilateral structure located beneath the surface of the temporal lobe known to play a pivotal role in the formation and stabilization of memories. \\
Name derived from the shape of the structure in comparison to a seahorse with the genus name \textit{Hippocampus}. \\
Neuroplastic changes are most robust during the early years of life. Brain most susceptible to wiring changes in infancy, childhood, adolescence, and early adulthood. Most powerful experiences in one's life. \\
\textbf{Neurogenesis:} growth and branching of axons and dendrites, sprouting of dendritic spines, synapse formation and strentgthening, synapse pruning and elimination, glial cell formation and differentiation, and axonal myelination are all continuing at robust rates during childhood. \\
Myelination of axons interconnecting cells within the cerebral cortex continues until past twenty years of age. \\
\bigskip 

\begin{center}
    Value this time of brain growth, provide good parenting and a loving atmostphere for childrens childhoods. -Presti
\end{center}

\end{document}
