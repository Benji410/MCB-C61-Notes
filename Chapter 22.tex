\documentclass{article}
\usepackage[utf8]{inputenc}
\usepackage{graphicx}

\title{Mind, Consciousness, and Reality}
\author{MCB C61 with Professor David Presti \\ \\ Benjamin Lee}
% \date{8 March 2018}

\begin{document}

\maketitle

\textbf{Key Concepts:}
\begin{itemize}
    \item Intelligence
    \item Artificial intelligence
    \item Reductionism
    \item Quantum measurement problem
    \item Origin of Life
    \item SETI
    \item Physicalism
    \item Explanatory Gap, hard problem
    \item Revolution in science
\end{itemize}

\newpage
\section{Intelligence}
\begin{itemize}
    \item From Latin \textit{intelligentum} meaning to discern or comprehend
    \item Ability to acquire and retain knowledge, learn, grasp truths and patterns, reason, and apply all this toward solving problems
    \item \textbf{Understanding:} capacity to discern relationships and appreciate connections
\end{itemize}

No mention of mind, mentality, or consciousness, meaning intelligence does not include awareness, inner subjective experience, or sentience.

\section{Artificial Intelligence}
By the previous definitions of intelligence, we can say machines have become quite intelligent. 

\begin{itemize}
    \item Chess games, Jeopardy have been mastered by machines already and have beaten even the greatest of human opponents. 
    \item Machines can be programmed to create music and art, and are becoming increasingly sophisticated as time goes on
\end{itemize}

\noindent \textbf{Alan Turing (1912 - 1954)}
\begin{itemize}
    \item Formed the basis of computational ideas at the dawn of modern thinking about computability in 1940s and 1950s
    \item Computers speed and capacity have grown a billionfold since then, but same principles
    \item Some say intelligence of computers will soon exceed us humans some day
    \item Known as \textit{technological singularity}
\end{itemize}

However, computers still lack minds. No matter how fast they compute, intelligent and creative they are, or how fast they learn, they still do not have mental experience, conscious awareness, sentience, etc. \\

Max Delbruck commented in 1949: "Any living cell carries with it the  experience of a billion years of experimentation by its ancestors. You cannot expect to explain so wise an old bird in a few simple words" \\

\section{Reductionism}
\textbf{Definition:} the practice of analyzing and describing a complex phenomenon in terms of phenomena that are held to represent a simpler or more fundamental level, especially when this is said to provide a sufficient explanation. \\

Ex: Biology is explained in terms of what is considered to be the more basic science of chemistry, and chemistry is understood in terms of the fundamental rules of matter and energy as described by physics, and physics is grounded in elegant mathematical structures and equations. \\ 

\begin{itemize}
    \item Powerful framework for describing, explaining, and understanding our world
    \item Much can be calculated, predicted engineered and constructed
    \item Applications are legion, from exploration of deep space, to the gizmos of technology that have become part of our daily lives, to molecular manipulations helpful in the treatment of human disease
\end{itemize}

\section{Quantum Measurement Problem}
We have this great way of explaining how things in our world work and we think we have a grasp on it, but we could also argue that we don't really know much at all.  

\begin{itemize}
    \item Periodic table has all these properties of all matter we come into contact with
    \item Principles of quantum physics applied to the internal structures of the protons and neutrons of the atomic nucleus yield fundamental particles of nature: quarks, gluons, electrons, muons, neutrinos, bosons, photons, and taus
    \item All this success only accounts for 4\% of matter/energy in the universe
    \item 96\% is composed of dark matter and dark energy, things we don't understand
\end{itemize}

\textbf{"Measurement Problem" of Quantum Physics:} 
\begin{itemize}
    \item Particles can exist in multiple states and places simultaneously, governed by quantum wave function
    \item Particle said to exist as superposition of multiple alternative possibilities
    \item Yet we perceive reality as actualities and not potentialities
    \item \textbf{"reduction" or "collapse"}: transition from superposition of potentialities to discrete values. 
    \item Defines the connection between the microscopic quantum world and macroscopic classical world
    \item This "reduction" or "collapse" is the measurement problem because we don't know how it happens 
\end{itemize}

\textbf{Other topics in book:}
\begin{itemize}
    \item Nonlocal entanglement: particles that are separated by arbitrarily large distances may still influence one another, instantaneously
    \item String theories explaining multiple dimensions above 4 
    \item Mathematics offers the point of ultimate guide to what we regard to be truths about physical reality. zero to infinity, real and imaginary, discrete and contiuous (Plato  "ideal forms") 
\end{itemize}

\section{Origin of Life}
\begin{itemize}
    \item "Big Bang": 13.7 billion years ago setting in motion processes that formed our galaxy, sun, and solar system
    \item Earth is right distance to produce conditions conducive to the presenve of life
    \item Elexir of life, good temperature, proteins, nucleic acids, etc
    \item How did life originate?
    \item Guess: small RNA molecules formed in the primordial soup and that such molecules were capable of replicating and catalyzing chemical reactions
    \item Another theory: life was droppepd off on Earth from a comet
    \item 
\end{itemize}

\subsection{Search for Extraterrestrial Intelligence (SETI)}
\begin{itemize}
    \item Program has been going strong 
    \item Search the heavens for signs of radio signal in cosmic noise
\end{itemize}

\subsection{Origin in Astronomy}
\begin{itemize}
    \item Nicolaus Copernicus (1473-1543) seeded idea
    \item Galileo Galilei (1564-1642) staked out th eterritory
    \item Isaac Newton (1643-1727) locked it with mathematical description of both celestial and terrestrial motion
    \item Mathematical foundations of Newton applied to phenomenon such as light, electricity, magnetism, gaseous pressure, temperature, thermodynamics, etc. 
\end{itemize}

\noindent \textbf{Physicalism:} describes larger context within we interpret scientific analyses, the stage or landscape upon which our inquiry is carried out. Metaphysical framework. \\

Must define mind, mental experience, and consciousness in terms of properties of matter. Manifestation of mind-body problem \\

\noindent \textbf{Hard problem (of consciousness):} How subjectivity is related to the physical workings of the brain and body. Also referred to as the \textbf{explanatory gap} between physical and subjective \\ 

\noindent \textbf{Approaches to mind-body problem:}
\begin{itemize}
    \item DUalism: two separate domains - matter(physcial stuff) and the mind (mental stuff) - that come together and interact within the body
    \item Mentalism (Idealism): What is real is the mental domain and somehow our experience of what we call physical is derived from that
    \item Panpsychism: mind and consciousness are everywhere, fundamental to the structure of reality
\end{itemize}

\section{Revolution of Science}
Several trajectories of investigating the conscious experience and mind-body problem
\begin{enumerate}
    \item Continued direct investigation fo the brain, body, and life, suing and extending the ideas and methods already in play
    \item Refined analysis of mental experience, drawing from tools of contemplative tradition, and in particular the current dialogue between science and Buddhism
    \item Radically empirical extensions of research in biology and psychology, honoring all the data, no matter how weird, especially if it bears directly on the mind-matter connection
    \item appreciating that the next truly big scientific revolution may involve deep connection between fundamental physics and consciousness, so look for opportunities to investigate such connections
\end{enumerate}

Read more indepth explanations in book :)



\end{document}