\documentclass{article}
\usepackage[utf8]{inputenc}
\usepackage{graphicx}

\title{Imaging the Brain}
\author{MCB C61 with Professor David Presti \\ \\ Benjamin Lee}
% \date{8 March 2018}

\begin{document}

\maketitle

\textbf{Key Concepts:}
\begin{itemize}
    \item Lesion: stroke, tumor, traumatic injury, disease
    \item Static / structural brain imaging
    \item X-ray imaging
    \item Computed axial tomography (CAT, CT)
    \item Magnetic resonance imaging (MRI)
    \item Nuclear spin, NMR
    \item Magnetic Field Strength: Gauss, Tesla
    \item Dynamic / Functional brain imaging
    \item Electroencephalography (EEG)
    \item Hans Berger
    \item Wilder Penfield, Epileptogenic tissue
    \item Electrocorticography (ECoG) 
    \item Magnetoencephalography(MEG) 
    \item Positron emission tomography (PET) 
    \item Positron Emission, annilation, gamma photons
    \item Cyclotron
    \item Ernest Lawrence
    \item Transuranium Elements
    \item "Dark Energy" and the brain
    \item fMRI, hemoglobin, BOLD signal
    \item temporal and spatial resolution
\end{itemize}

\newpage

\section{Lesions}

The general term used to denote an injury to, or abnormality in, the body. \\

Among the causes of brain lesions are \textbf{stroke, tumor, physical trauma, and certain brain diseases}. \\

\begin{itemize}
    \item Stroke: occurs when there is a disturbance in blood flow to a region of the brain sufficient to produce a loss of function. 
    \begin{itemize}
        \item Lasts more than a few minutes, then there is likely to be cell death and loss of function may be permanent
        \item Two Causes: \textbf{blockage of blood flow} and \textbf{hemorrhage}
        \item Blockage occurs when particulate matter (blood clot or atherosclerotic plague) becomes lodged in a blood vessel
        \item Hemorrhage occurs when a blood vessel breaks and leaks into the surrounding tissue. Occurs at weakness in blood vessel structure. 
        \item Aneurysm: when structural weakness of blood vessel bulges out from  pressure of the blood
        \item Hemorrhagic stroke: when an aneurysm ruptures
    \end{itemize}
    
    \item Brain Tumor: anamolous, abnormal proliferation of cell in the brain
    \begin{itemize}
        \item Anomalous growth myay be either benign (nonspreading) or malignant(able to metastasize and spread) 
        \item Abnormal tissue growth often disturbs the normal functioning of nearby neural tissue, producing symptoms that may manifest as changes in perception, other mental function or behavior. 
    \end{itemize}
    
    \item Physical Trauma: resulting from injury to the head
    \begin{itemize}
        \item Two general types: closed or penetrating
        \item Closed: head injury occurs when there is a whack to the head, or sudden powerful acceleration or deceleration. Integrity of the skull is not broken and brain is not penetrated
            \subitem Concussion: damage to the brain occurring from shearing forces within the tissue or internal contact with the bone. (internal swelling can occur) 
        \item Penetrating: integrity of the skull is compromised and the brian comes into direct contact with an external agent of damage
    \end{itemize}
    
    \newpage 
    \item Disease
    \begin{itemize}
        \item Best-studied example: 
        \item Parkinson's disease: a neurodegenerative condition characterized by slowness and difficulty with movement.
        \subitem Associated with neuronal death in a specific region of the brain: the \textbf{substantia nigra}, one of the clusters of cells in the brain stem that uses dopamine as a neurotransmitter. 
    \end{itemize}
\end{itemize}

\section{Brain Imaging}
To make connections between regions of the brain and specific functions, it is necessary to identify as precisely as possible the location of a brain lesion and then match that with symptoms exhibited by the individual who has the lesion. \\
Historically, only able to locate lesions in the brain after death. 

\subsection{X-ray}
\begin{itemize}
    \item Kind of electromagnetic radiation having energy substantially higher than that of visible light or ultraviolet light. 
    \item Able to visualize the internal structure of living bodies
    \item Wilhelm Rontgen (1845-1923): described X-radiation and received the very first Nobel Prize in Physics in 1901 in honor of his discovery. 
    \item X-rays can penetrate solid matters because of high energy. 
    \item Used to detect skeletal structure because bone is less permeable to X-rays
    \item Later developed x-ray to visualize different organ tissues with x-ray photograph (brain lesions)
\end{itemize}

\subsection{Computed Axial Tomography (CAT, CT)}
X-ray photograph may permit a brain lesion to be seen, but limited in location \\ 
Precision can be obtained by taking a series of x-rays from different angles and constructing a 3-D image of the brain\\

Obtained in 1960's due to more powerful computers 
\begin{itemize}
    \item Result: CT or CAT scan
    \item \textbf{Computed} (using a computer), \textbf{Axial} (slices are along a central axis of symmetry of the brain), \textbf{Tomography} (making a series of images, essentially of slices of the brain)
    \item Sophisticated x-ray imaging process that generates a three-dimensional representation of the brain's internal structure
    \item Used to aid in diagnosis in clinical medicine, can also visualize other parts
    \item However, x-rays damage molecules with their high energy. 
        \subitem Break covalent chemical bonds, disrupting structure
        \subitem Proteins or lipids likely to be permanently damaged and function destroyed. 
        \subitem DNA damage may be repairable by enzymes that fix DNA, but may change nucleotide sequence of DNA causing mutations in genes and other anomalous activity in gene transcription.
        \subitem X-radiation is toxic and exposure needs to be limited to maintain health. 
\end{itemize}


\subsection{Magnetic Resonance Imaging (MRI)}
Introduced in the 1980's \\
Based off a physical phenomenon called quantum spin. property of subatomic properties (protons and neutrons). \\
Atomic nuclei possess a nuclear spin that arises form the combination of the spins of the constituent protons and neutrons \\
Interacts with magnetic fields, subatomic particle will align its spin with an imposed magnetic field, analogous to a compass needle aligning in Earth's magnetic field. 
\begin{itemize}
    \item Can produce a three-dimensional reconstruction of the internal structure of a living brain or other parts of the body
    \item Uses quantum spin
\end{itemize}

\subsection{Nuclear Magnetic Resonance (NMR)}
1940's tech was developed to measure NMR. \\
NMR spectrometer consists of a large magnet to produce a very strong magnetic field, and a device to generate electromagnetic radiation of appropriate energy to perturb the alignment of nuclear spins \\
Magnets that are several teslas in strength, the corresponding energies of perturbing frequency are typically in the radio-frequency region of the electromagnetic spectrum \\

\textbf{Nikola Tesla (1856-1943)}\\
Inventor, engineer and wizard of electricity \\ 
Tesla is a unit of magnetic field strength \\

\textbf{Carl Friedrich Gauss(1777 - 1855)} \\
Mathematician and physicist \\ 
Gauss is another unit of magnetic field strength 

\textbf{Magnetic Field Strengths:}
\begin{itemize}
    \item 1 tesla \equiv 10,000 gauss
    \item Earth's magnetic field measures 0.5 gauss, or 50 microteslas
    \item Small magnets (refrigerator magnets) about 50 gauss, or 5 milliteslas 
    \item Magnet strength of NMR spectrometers are substantial, beyond Earth's 
\end{itemize}

\textbf{How NMR works:}
\begin{itemize}
    \item Nuclear spins of various atoms in a organic molecule will align in a strong magnetic field
    \item Alignment can be perturbed when the right radio-frequency energy of electromagnetic radiation is absorbed
    \item The most abundant atom in organic molecules is hydrogen; different hydrogen atoms in a given molecule will have different resonant frequencies, depending on electromagnetic environment
    \item NMR can be used to help determine unknown molecular structures of organic molecules, by looking at the energies needed to perturb the alignment of the spins 
    \item NMR spectroscopy in organic chemistry
\end{itemize}

\textbf{How NMR can construct images of living organisms}
\begin{itemize}
    \item Living organisms are mostly made of hydrogen atoms from water molecules. 
    \item Possible to focus on different spin properties of hydrogen atoms in the body
    \item Different tissues will form different chemical environments for water. 
    \item Spatial patterns of different resonant energies can be used to construct an image of the interior of an organism (brain) 
    \item Magnetic fields and radio waves penetrate human body easily
\end{itemize}



% \underline{Lecture Notes:} \\
% Dolphin brains are as sophisticated as ours 
% coronal or frontal section
% horizontal or axial section
% sagittal section

% injury to the brain = lesion

% causes of circumscribed / localized lesions
% stroke, tumor, traumatic injury and certain brain diseases, 

% static/structural brain imaging
% visualizing anatomical structure and locating lesions
% dissection, xray, ct scan, mri

% xray fitting test, "scientific shoe fitting" using x rays in 1920s and 1950s

% nuclear magnetic resonance (NMR) 
% nuclear spin
% in a magnetic field, spin can align in two different energy states. (up and down)
% can flip from one state to another

% magnetic field strength 
% 1 tesla = 10,000 gauss
% car Freidrich Gauss (1777 - 1855)
% Nikola Tesla (1856- something) 

% geomagnetic field ~0.5 gauss
% = 0.5 * 10 ^-4 tesla
% = 50 * 10 ^-6 tesla 
% % 3 tesla = 60,000x geomagnetic field
% Wilder Penfield brain mapping using electrical sitimulation and recording
% Electro corticography(ecog)

% seizure 
% run away neural activity

% eeg (electroencephalography) Brain waves
% delta: < 4hz
% theta: 4-8 hz
% alpha: 8-15 hz
% beta: 15-30 hz
% gamma: >30hz

% signals happen because of action potentials firing constantly causing voltage flows around the brain generating magnetic fields. 

% Hans Berger (1873-1941)
% 1920s first measurement of human eeg
% 1892 near-death accident: discovered during military maneuver when a canon almost crushed him. Got message from sister saying she felt something. thought there might be some connection

% MEG (magnetoencephalography)
% using SQUID technology (Superconducting quantum interference device). 

% Functional magnetic resonance imaging (fmri) scanner

% positron emission tomography (PET)



\end{document}