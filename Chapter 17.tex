\documentclass{article}
\usepackage[utf8]{inputenc}
\usepackage{graphicx}

\title{Imaging the Brain}
\author{MCB C61 with Professor David Presti \\ \\ Benjamin Lee}
% \date{8 March 2018}

\begin{document}

\maketitle

\textbf{Key Concepts:}
\begin{itemize}
    \item Lesion: stroke, tumor, traumatic injury, disease
    \item Static / structural brain imaging
    \item X-ray imaging
    \item Computed axial tomography (CAT, CT)
    \item Magnetic resonance imaging (MRI)
    \item Nuclear spin, NMR
    \item Magnetic Field Strength: Gauss, Tesla
    \item Dynamic / Functional brain imaging
    \item Electroencephalography (EEG)
    \item Hans Berger
    \item Wilder Penfield, Epileptogenic tissue
    \item Electrocorticography (ECoG) 
    \item Magnetoencephalography(MEG) 
    \item Positron emission tomography (PET) 
    \item Positron Emission, annihilation, gamma photons
    \item Cyclotron
    \item Ernest Lawrence
    \item Transuranium Elements
    \item "Dark Energy" and the brain
    \item fMRI, hemoglobin, BOLD signal
    \item temporal and spatial resolution
\end{itemize}

\newpage

\section{Lesions}

The general term used to denote an injury to, or abnormality in, the body. \\

Among the causes of brain lesions are \textbf{stroke, tumor, physical trauma, and certain brain diseases}. \\

\begin{itemize}
    \item Stroke: occurs when there is a disturbance in blood flow to a region of the brain sufficient to produce a loss of function. 
    \begin{itemize}
        \item Lasts more than a few minutes, then there is likely to be cell death and loss of function may be permanent
        \item Two Causes: \textbf{blockage of blood flow} and \textbf{hemorrhage}
        \item Blockage occurs when particulate matter (blood clot or atherosclerotic plague) becomes lodged in a blood vessel
        \item Hemorrhage occurs when a blood vessel breaks and leaks into the surrounding tissue. Occurs at weakness in blood vessel structure. 
        \item Aneurysm: when structural weakness of blood vessel bulges out from  pressure of the blood
        \item Hemorrhagic stroke: when an aneurysm ruptures
    \end{itemize}
    
    \item Brain Tumor: anamolous, abnormal proliferation of cell in the brain
    \begin{itemize}
        \item Anomalous growth myay be either benign (nonspreading) or malignant(able to metastasize and spread) 
        \item Abnormal tissue growth often disturbs the normal functioning of nearby neural tissue, producing symptoms that may manifest as changes in perception, other mental function or behavior. 
    \end{itemize}
    
    \item Physical Trauma: resulting from injury to the head
    \begin{itemize}
        \item Two general types: closed or penetrating
        \item Closed: head injury occurs when there is a whack to the head, or sudden powerful acceleration or deceleration. Integrity of the skull is not broken and brain is not penetrated
            \subitem Concussion: damage to the brain occurring from shearing forces within the tissue or internal contact with the bone. (internal swelling can occur) 
        \item Penetrating: integrity of the skull is compromised and the brian comes into direct contact with an external agent of damage
    \end{itemize}
    
    \newpage 
    \item Disease
    \begin{itemize}
        \item Best-studied example: 
        \item Parkinson's disease: a neurodegenerative condition characterized by slowness and difficulty with movement.
        \subitem Associated with neuronal death in a specific region of the brain: the \textbf{substantia nigra}, one of the clusters of cells in the brain stem that uses dopamine as a neurotransmitter. 
    \end{itemize}
\end{itemize}

\section{Structural/Static Brain Imaging}
To make connections between regions of the brain and specific functions, it is necessary to identify as precisely as possible the location of a brain lesion and then match that with symptoms exhibited by the individual who has the lesion. \\
Historically, only able to locate lesions in the brain after death. 

\subsection{X-ray}
\begin{itemize}
    \item Kind of electromagnetic radiation having energy substantially higher than that of visible light or ultraviolet light. 
    \item Able to visualize the internal structure of living bodies
    \item Wilhelm Rontgen (1845-1923): described X-radiation and received the very first Nobel Prize in Physics in 1901 in honor of his discovery. 
    \item X-rays can penetrate solid matters because of high energy. 
    \item Used to detect skeletal structure because bone is less permeable to X-rays
    \item Later developed x-ray to visualize different organ tissues with x-ray photograph (brain lesions)
\end{itemize}

\subsection{Computed Axial Tomography (CAT, CT)}
X-ray photograph may permit a brain lesion to be seen, but limited in location \\ 
Precision can be obtained by taking a series of x-rays from different angles and constructing a 3-D image of the brain\\

Obtained in 1960's due to more powerful computers 
\begin{itemize}
    \item Result: CT or CAT scan
    \item \textbf{Computed} (using a computer), \textbf{Axial} (slices are along a central axis of symmetry of the brain), \textbf{Tomography} (making a series of images, essentially of slices of the brain)
    \item Sophisticated x-ray imaging process that generates a three-dimensional representation of the brain's internal structure
    \item Used to aid in diagnosis in clinical medicine, can also visualize other parts
    \item However, x-rays damage molecules with their high energy. 
        \subitem Break covalent chemical bonds, disrupting structure
        \subitem Proteins or lipids likely to be permanently damaged and function destroyed. 
        \subitem DNA damage may be repairable by enzymes that fix DNA, but may change nucleotide sequence of DNA causing mutations in genes and other anomalous activity in gene transcription.
        \subitem X-radiation is toxic and exposure needs to be limited to maintain health. 
\end{itemize}


\subsection{Magnetic Resonance Imaging (MRI)}
Introduced in the 1980's \\
Based off a physical phenomenon called quantum spin. property of subatomic properties (protons and neutrons). \\
Atomic nuclei possess a nuclear spin that arises form the combination of the spins of the constituent protons and neutrons \\
Interacts with magnetic fields, subatomic particle will align its spin with an imposed magnetic field, analogous to a compass needle aligning in Earth's magnetic field. 
\begin{itemize}
    \item Can produce a three-dimensional reconstruction of the internal structure of a living brain or other parts of the body
    \item Uses quantum spin
\end{itemize}

\subsubsection{Nuclear Magnetic Resonance (NMR)}
1940's tech was developed to measure NMR. \\
NMR spectrometer consists of a large magnet to produce a very strong magnetic field, and a device to generate electromagnetic radiation of appropriate energy to perturb the alignment of nuclear spins \\
Magnets that are several teslas in strength, the corresponding energies of perturbing frequency are typically in the radio-frequency region of the electromagnetic spectrum \\

\textbf{Nikola Tesla (1856-1943)}\\
Inventor, engineer and wizard of electricity \\ 
Tesla is a unit of magnetic field strength \\

\textbf{Carl Friedrich Gauss(1777 - 1855)} \\
Mathematician and physicist \\ 
Gauss is another unit of magnetic field strength \\

\newpage

\noindent \textbf{Magnetic Field Strengths:} 
\begin{itemize}
    \item 1 tesla $\equiv$ 10,000 gauss
    \item Earth's magnetic field measures 0.5 gauss, or 50 microteslas
    \item Small magnets (refrigerator magnets) about 50 gauss, or 5 milliteslas 
    \item Magnet strength of NMR spectrometers are substantial, beyond Earth's 
\end{itemize}
\bigskip
\nodindent \textbf{How NMR works:}
\begin{itemize}
    \item Nuclear spins of various atoms in a organic molecule will align in a strong magnetic field
    \item Alignment can be perturbed when the right radio-frequency energy of electromagnetic radiation is absorbed
    \item The most abundant atom in organic molecules is hydrogen; different hydrogen atoms in a given molecule will have different resonant frequencies, depending on electromagnetic environment
    \item NMR can be used to help determine unknown molecular structures of organic molecules, by looking at the energies needed to perturb the alignment of the spins 
    \item NMR spectroscopy in organic chemistry
\end{itemize}
\bigskip
\noindent \textbf{How NMR can construct images of living organisms}
\begin{itemize}
    \item Living organisms are mostly made of hydrogen atoms from water molecules. 
    \item Possible to focus on different spin properties of hydrogen atoms in the body
    \item Different tissues will form different chemical environments for water. 
    \item Spatial patterns of different resonant energies can be used to construct an image of the interior of an organism (brain) 
    \item Magnetic fields and radio waves penetrate human body easily
\end{itemize}

\subsubsection{NMR Imaging}
1980's introduced into hospitals to expand the imaging capacity beyond that of CT. \\
Does not involve exposure to toxic x-rays. Although strong magnetic exposure is still questioned \\

The word "nuclear" in NMR came under criticism because although it referred to atomic nucleus, people thought it inferred nuclear radiation or weapons. So N was dropped and became known as simply Magnetic Resonance Imaging. \\

MRI of brain yields a sharpness of anatomical detail superior to CT. 


\section{Functional/Dynamic Brain Imaging}
Static brain imaging was focused on locating lesions and studying the structure of the brain. \\
Functional or Dynamic Brain Imaging focuses on where functions relate to parts of the brain, such as "how does looking at an image relate to a certain part of the brain?" 

\subsection{Electroencephalogram (EEG)}
When nerve cells send and receive signals they create electric current flows  \\
This is caused by the movement of electric charge associated with the flow of ions in axons and dendrites. (action potentials) \\

\begin{itemize}
    \item Electrical charges generate electromagnetic fields that pass through surrounding tissue detectable through the head 
    \item Electrodes are attached to scalp to detect these fields 
    \item Electric field changes with origins in the neural activity of the brain can be recorded 
    \item Graph of brain electric field changes as a function of time (EEG). Referred to as brain wave sometimes
    \item Gives a measure of neural activity averaged over large regions of the cerebral cortex
    \item Electric fields are distorted and smooshed because of brain and skin conductivity. 
    \item Not very precise for locating neural  activity (bone, skin, and tissue in between) 
\end{itemize}

\begin{itemize}
    \item EEG can use from 2 nodes to 64-128 electrodes
    \item Terrible spatial resolution, but great time resolution
    \item EEG Frequencies of electrical oscillation:
        \begin{itemize}
            \item Delta $\delta$ (\< 4 Hz)
            \item Theta $\theta$ (4-7 Hz)
            \item Alpha $\alpha$ (8-15 Hz)
            \item Beta $\beta$ (6-30 Hz)
            \item Gamma $\gamma$ (\> 30 Hz)
        \end{itemize}
    \item Fourier analysis may be conducted on EEG data to elucidate the various frequency components that contribute to overall brain wave
\end{itemize}

\noindent \textbf{Hans Berger (1873 - 1941)} 
\begin{itemize}
    \item German physician who made first EEG recording and named it. 
    \item Military man who went to first studied math and astronomy at University of Berlin before enlisting in 1892. 
    \item Had an accident where he almost got crushed by an artillery gun and was later contacted by his father that his close sister felt something terrible happened to Hans 
    \item Switched to study of medicine to investigate brain and relation so mind. (wanted to uncover sister's presumed telepathic experience. \\
\end{itemize}


\noindent \textbf{Wilder Penfield} 
\begin{itemize}
    \item 1940's and 1950's, recorded electrical activity directly from the cerebral cortex of people who had their brain exposed during surgeries
    \item Allowed for higher resolution of the locations of neural activity
    \item Mapped regions that determined the relationship between regions of the cortex and various functions, such as sensory perception, muscle movement, language generation, and language comprehension (originally interested in mapping locations of seizures)  
\end{itemize}

\subsection{Electrocorticography (ECoG)}
This direct brain recording has continued to be developed into a highly refined technique called \textbf{electrocorticography.} 

\begin{itemize}
    \item Technique used before brain surgery in which epileptogenic tissue it to be removed 
    \item While epileptogenic tissue is being mapped with an \textit{array} of ECoG electrodes, neurosurgeons and scientists may collaborate to address interesting question about the function and dynamic activity of the cortex
\end{itemize}

\subsection{Magnetoencephalograph (MEG)}
\begin{itemize}
    \item Electric currents generate magnetic fields
    \item Magnetic fields induced by the electric currents associated with neural activity can be measured in the vicinity of the head
    \item Results in a brain wave called a \textbf{magnetoencephalogram}, named by analogy to electroencephalogram
    \item Measured with MEG
\end{itemize}

\noindent \textbf{Magnetic fields in brain:} 
\begin{itemize}
    \item Magnetic fields in human brain measure 1 picotesla (10$^{-12}$) at the surface of skull
    \item Earth's magnetic field is approx. 50 microteslas (50 x 10$^{-6}$) (50 million times stronger)
    \item Ambient magnetic noise in urban environment range of 0.1 microtesla (10$^{-7}$) (about 100,000 times stronger than brain magnetic fields) 
\end{itemize}

\noindent \textbf{How MEG works?}
\begin{itemize}
    \item Two primary hurdles: constructing very sensitive detectors of magnetic fields and shielding the whole measurement process from ambient electromagnetic noise
    \item Shielding: accomplished by building rooms made of metals resistant to penetration from magnetic fields, and use of electronic systems to measure and cancel sources of magnetic noise. (analogous to noise canceling headphones) 
    \item Detection of magnetic fields: MEG uses \textbf{SQUID} or \textbf{S}uperconducting \textbf{Q}uantum \textbf{I}nterference \textbf{D}evices
    \subitem Based on properties of superconducting currents in the presence of magnetic fields
    \item Very expensive and challenging to reform images made by MEG. Applications limited
\end{itemize}

\subsection{Positron Emission Topography (PET)}
\begin{itemize}
    \item Uses properties of particular radioactive chemicals to visualize cellular activity in the brain
    \item Radioactivity, radiation, radiate, radiant, ray. To shine, emit, glow, extend from center, radius
    \item \textbf{Radioactive decay} is when an atom has a combination of protons and neutrons that are unstable and will transform by emitting a high-energy particle of one kind or another.
    \item Radioactivity is the name given to the high-energy particles emitted by atoms during the decay process
    \item Several decay processes: 
        \begin{itemize}
            \item \textbf{alpha decay:} when an unstable nucleus emits a blob made of two protons and two neutrons
            \item \textbf{beta decay:} when an unstable nucleus emits either an electron or a positron
            \item \textbf{gamma decay:} when an unstable nucleus emits a gamma-ray photon
        \end{itemize}
    \item PET uses unstable elements that undergo beta decay by emitting a positron
        \subitem result is a new element that is the same atomic mass but is one unit lower in atomic number. Because loss of positive charge in the form of a positron (e$^+$) effectively converts one of the nuclear protons into a neutron
    \item PET uses radioactive forms of the elements carbon, oxygen and fluorine. \\ Isotopes:
        \begin{itemize}
            \item carbon-11 $\rightarrow$ boron-11 + e$^+$ (half-life ~ 20 minutes)
            \item oxygen-15 $\rightarrow$ nitrogen-15 + e$^+$ (half-life ~ 2 minutes)
            \item fluorine-18 $\rightarrow$ oxygen-18 + e$^+$ (half-life ~ 110 minutes) 
        \end{itemize}
        \subitem Numbers give atomic masses of the isotopes. 
        \subitem EX: carbon-11 has 6 protons because carbon is defined as having 6 protons and 11 minus 6 = 5 neutrons. Most common isotope of carbon is carbon-12 because it is stable 6 protons and 6 neutrons. Carbon-12 does not decay. 
    \item Uses a fundamental property of the positron called antimatter particle corresponding to the electron. 
    \item When a particle encounters its corresponding antiparticle, there is a complete \textbf{annihilation} of the mass of both particles, converting all the mass into energy ($E = mc^2$)
        \subitem When a positron is emitted via radioactive decay, it will very quickly encounter an electron and annihilate
    \item Resulting energy of this positron-electron annihilation emerges as \textbf{two high-energy gamma-ray photons} flying off in exactly opposite directions. 
    \item Placing gamma-ray detectors all around the region where the radioactive material is located and watching for photons emerging simultaneously and exactly 180 degrees apart, it's possible (triangulation), to precisely determine the source of the radioactive decay 
\end{itemize}

\subsubsection{Radioactive Isotopes and how they work with PET}
\noindent \textbf{Fluorine-18}
\begin{itemize}
    \item First isotope used for PET
    \item Radioactive variant of glucose made by replacing one of the oxygen atoms in the sugar with fluorine-18
    \item All cells use glucose as an energy source (transported via blood)
    \item Cells that are working harder use more glucose
    \item Robustly, nerve cells generating signals in the form of action potentials require more glucose to make the ATP needed to operate their Na/K pumps and maintain membrane potential (more active, more glucose it takes) 
    \item Glucose radioactively labeled with fluorine-18 is injected into a person's circulatory system, it will flow throughout and be absorbed into cells just as normal glucose
    \item However, it isn't processed as glucose to energy because it is recognized as an imposter, so it \textbf{accumulates} in the cells becoming a radioactive hot spot
    \item Can locate greatest neural activity by detecting radioactive hot spots 
    \item PET scanner consists of an array of many detectors designed to measure gamma rays emitted by isotope
\end{itemize}

\noindent \textbf{Oxygen-15}
\begin{itemize}
    \item PET scan method uses radioactive water (H$_2$O) where O-15 decays by positron emission
    \item Radioactive water is injected into the subject's blood 
    \item Regions generating more neural signals require more energy, so more blood flows as well as the radioactive water flowing within the blood
    \item This can be measured in the PET scanner detecting the positron decay from O-15
\end{itemize}

\newpage
\noindent \textbf{Carbon-11}
\begin{itemize}
    \item Suppose one wants to measure the locations of dopamine receptors in a brain 
    \item Choose a molecule that sticks to dopamine receptors (either agonist or antagonist)
    \item Replace molecules carbon with C-11 and inject into person's bloodstream
    \item The molecule will cross the Blood-Brain Barrier and accumulate on dopamine receptors 
    \item Distribution of receptors in the brain can be located by radioactive decay with PET
\end{itemize}

\subsubsection{Cyclotron}
Positron-emitting radioactive isotopes have short half-lives, ranging from 2 minutes to 2 hours \\

\noindent Cyclotron is able to make the isotopes on the spot and quickly incorporate into desired molecular form to be administered for PET measurement \\

\noindent \textbf{Ernest Lawrence (1901 - 1958)}\\
Cyclotron made in UC Berkeley in 1930 by Ernest 

\begin{itemize}
    \item Cyclotron accelerates charged particles (like atomic nuclei) to very high velocities using oscillating electromagnetic fields
    \item Imparts energy to the charged particle as they move around in a circular path
    \item Speeding particles can be smashed into other atoms, and sometimes will stick together and new atoms will be formed
\end{itemize}

\noindent \textbf{Fun Fact:} 1940's: UC Berkeley physicists discovered that cyclotrons could be used to produce heavier chemical elements never seen before in nature \\

Bombarded uranium (atomic number 92),the heaviest chemical known to exist, with protons, deuterons and alpha particles at high velocities in cyclotron\\

Elements neptunium (93), plutonium (94), americium (95), curium, (96) berkelium (97), californium (98), einsteinium (99), fermium (100), and mendelevium (101) were all discovered at UC Berkeley with this process \\

\noindent PET scan is a complex and expensive process and is highly invasive due to toxic radioactive exposure. Limited ways to use because of these reasons. \\

\indent Brain is working hard all the time, due to PET scan using radioactive glucose to measure energy consumption. We don't know what all this activity is for and has been dubbed as \textbf{"Dark energy"} for the mystery. 

\subsection{Functional Magnetic Resonance Imaging (fMRI)}
\begin{itemize}
    \item Same technique as MRI technologies, but collect a series of MRIs over time and looking at change in neural activity. 
    \item Cells derive most of their energy from breakdown of glucose using oxygen
    \item Hemoglobin is the oxygen-carrying protein in red blood cells and produces a different \textbf{magnetic perturbation} effect on its local environment depending on whether oxygen is attached to it or not
    \item This change can be monitored by looking at magnetic resonance signal of hydrogen atoms in H$_2$O molecules in the vicinity of hemoglobin molecules in blood 
    \item Change captured in parameter \textbf{BOLD} signal (blood-oxygen-level dependence), measures the increased flow of blood into regions of the brain that are more neutrally active
    \item Good spatial resolution (within 1 millimeter) 
    \item Time resolution is several seconds
    \item EEG faster time (milliseconds) but poor spatial
\end{itemize}

\begin{tabular}{||c|c|c||}
    \hline
     \textbf{Imaging method} & \textbf{Spatial resolution} & \textbf{Temporal resolution}  \\ 
     \hline 
     EEG & several cm & milliseconds   \\
     MEG & mm & milliseconds \\
     fMRI & mm & seconds \\
     PET & cm & seconds to minutes \\
     \hline
\end{tabular}

\newpage 
\underline{Lecture Notes:} \\
Dolphin brains are as sophisticated as ours 
coronal or frontal section
horizontal or axial section
sagittal section

injury to the brain = lesion

causes of circumscribed / localized lesions
stroke, tumor, traumatic injury and certain brain diseases, 

static/structural brain imaging
visualizing anatomical structure and locating lesions
dissection, xray, ct scan, mri

xray fitting test, "scientific shoe fitting" using x rays in 1920s and 1950s

nuclear magnetic resonance (NMR) 
nuclear spin
in a magnetic field, spin can align in two different energy states. (up and down)
can flip from one state to another

magnetic field strength 
1 tesla = 10,000 gauss
car Freidrich Gauss (1777 - 1855)
Nikola Tesla (1856- something) 

geomagnetic field ~0.5 gauss
= 0.5 * 10 ^-4 tesla
= 50 * 10 ^-6 tesla 
% 3 tesla = 60,000x geomagnetic field
Wilder Penfield brain mapping using electrical sitimulation and recording
Electro corticography(ecog)

seizure 
run away neural activity

eeg (electroencephalography) Brain waves
delta: < 4hz
theta: 4-8 hz
alpha: 8-15 hz
beta: 15-30 hz
gamma: >30hz

signals happen because of action potentials firing constantly causing voltage flows around the brain generating magnetic fields. 

Hans Berger (1873-1941)
1920s first measurement of human eeg
1892 near-death accident: discovered during military maneuver when a canon almost crushed him. Got message from sister saying she felt something. thought there might be some connection

MEG (magnetoencephalography)
using SQUID technology (Superconducting quantum interference device). 

Functional magnetic resonance imaging (fmri) scanner

positron emission tomography (PET)



\end{document}