\documentclass{article}
\usepackage[utf8]{inputenc}
%\usepackage{graphicx} try adding graphics to notes

\title{Tongue and Taste}
\author{MCB C61 with Professor David Presti \\ \\ Benjamin Lee}
\date{13 March 2018}

\begin{document}

\maketitle

George Berkeley (1685 - 1753) \\
A Treatise Concerning the Principles of Human Knowledge (1710) \\
All we know is our perception, the entire world is created by our perception. \\
Coming to the limits of what we can say about human consciousness. \\
spent last lecture talking about olfactory sensors/bulbs \\

Stem cells and gustatory cell replacement

FLAVOR: 
    Taste
    Smell
    Pungency
    Texture
    
Taste receptor cell types: 
    believed that ion channels have something to do with taste
    salt
        ion channel sensitive to Na$^+$    
    sour
        ion channel sensitive to H$^+$ (acids taste sour)
    bitter
        30 GPCRs, encode for bitter taste. 
        Many different molecules (different shapes) found in nature taste bitter 
        (There is only one salt (i think) but many things can taste bitter. various bitter plant alkaloids: 
            caffeine
            quinine
            cocaine
            absinthin (absinth alcohol must get some)
    sweet
        2 GPCRs encode for sweet taste
        the molecules for sweet stuff have similar encodings and shapes
        GPCR dimer (two gpcr smashed together, with either the same gpcrs or different gpcrs
        gpcr multimers are a thing as well. 
        "sugar"
            sucrose (C, O, H hydrophilic)
            glucose
            fructose
            originated in SE Asia and South Pacific and reached Mediterranean in 1000BC
            Honey (original sugar source) as well as fruits
            saccharin (500x) (man made) (synthetic and artificial)
            aspartame (180x) (man made) (synthetic and artificial)
            stevioside (300x sweeter than sucrose) from stevia rebaudiana
            Max Delbruck (1906 - 1981) - Principle of Limited Sloppiness 1950 (how saccharin and aspartame were discovered)
            sucralose (600x) with sucrose, 3 OH groups replaced with Cl (splenda)
                2004 -2007 (equal sues splenda company, claiming false advertising: "there is no sugar in splenda and splenda's sweet taste does not come from sugar)
            Neotame (10000x) (how diet drinks are made. Little sugar molecules make calories almost 0) (taken from aspartame) approved by FDA in 2002
        Artificial sugars no good for you
        Miracle Fruit (synsepalum dulcificum) West Africa
            miraculin protein 199 amino acids
            binds to sweet reeptor gpcr dimer
            makes sour things taste sweet
            strong agonist effects only at highly acidic pH
    umami 
        "delicious", "savory", glutamate related to this savory taste
        GPCR, similar to the ones in our brains (glutamate receptors)
        related to cooked meats and mushrooms anything that has the burst of umami
        discovered in 1909 by Kikunae Ikeda of Imperial University of Tokyo, Japan
        Americans and Europeans continued to speak of 4 tastes until the 1990s
        umami taste receptor cloned in 2000
    Discussion of taste receptor for fats
    
    Chili peppers (capsicum annum) South America
        Cayenne
        Bell pepper (same species as cayenne)
            pepper: after asian pepper
        pasilla
        habenero (capsicum chinense)
        concentrations of capsaicin ("hot" substance in chili)
            why hot?
            capsaicin receptor a heat-activated ion channel in the pain pathway
            open Ca$^++$ channel, depolarization
            signals temperature hot, pain
            
            
Microvilli containing taste proteins
 
    




\end{document}